\documentclass[letterpaper]{report}
\usepackage[margin=1in]{geometry}
\parindent 0pt
\parskip 1em
\begin{document}
{\large \textbf{CPSC 304} 2015W2 \\ \textbf{Project Part II}}

\section*{Group Members}
\begin{tabular}{l l l l}
	\hline\hline\noalign{\vskip 1ex}
	\textbf{Name} & \textbf{Student No.} & \textbf{Unix ID} & \textbf{Email} \\ \noalign{\vskip .4ex}\hline\noalign{\vskip .8ex}
	Albert Xing & 40640104 & z6k8 & albert.xing@alumni.ubc.ca \\
	Calvin Cheng & 36090132 & o7x8 & calvin.cheng@alumni.ubc.ca \\
	Kyle Stadnyk & 52749025 & b5p5 & b5p5@ugrad.cs.ubc.ca \\
	\noalign{\vskip 1ex}\hline
\end{tabular}

By typing our names and student numbers in the above table, we certify that the work in the attached assignment was performed solely by those whose names and student IDs are included above.

In addition, we indicate that we are fully aware of the hrules and consequences of plagiarism, as set forth by the Department of Computer Science and the University of British Columbia.

{\bf We realize that in using a platform unsupported by the course and course staff we assume full responsibilities for all technical issues that may arise due to our choice of platform.}

\section*{Background}
We are building a financial portfolio managment application. In general, this application will help investors and financial advisors track and manage their portfolios.
Specifically, each portfolio is composed of any number of stock holdings.

\section*{Platform}
We will use Ruby on Rails for our application. The application will be deployed and hosted through Heroku. Heroku uses PostgreSQL as the backing DB for Rails.
The frontend will be built using HTML/CSS/JS. Frontend frameworks or libraries (React, Backbone, D3, jQuery) may be used.

\section*{Functionality}
There will be three user groups that will have different roles in the application:
\begin{description}
	\item[Investors] Investors, or clients, will be able to monitor and make changes to their portfolios. They will also be able to view and update their personal information.
	\item[Advisors] Financial advisors can manage portfolios on behalf of investors. They will be able to monitor and make changes to portfolios they are hired to manage. They will also be able to view their clients' contact information.
	\item[Administrators] System administrators will have complete read and write access to all information in the system.
\end{description}

Specifically, updates to portolios will be executed as {\em transactions}. Each transaction is either a {\em buy} or {\em sell}. Transactions will affect the holdings in the portfolio - either by adding a new holding, or updating an existing holding. Other updates to portfolios include adding or withdrawing cash.

When an investor or advisor wishes to view a portfolio, multiple database queries are needed. The first query must retrieve all portfolios associated with an investor. For example, to list portfolios owned by investor `1234': \verb|SELECT * FROM Portfolio WHERE owner_id = '1234';|

\end{document}
